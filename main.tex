\documentclass[a4paper,papersize,11pt]{jsarticle}

\usepackage{textcomp}
\usepackage[dvipdfmx]{graphicx}
%\usepackage{otf}
\usepackage{amsmath}

\pagestyle{empty}
%\setlength{\topmargin}{-72pt}
\setlength{\headheight}{0truecm}
\setlength{\headsep}{0truecm}
\setlength{\footskip}{0truecm}

\newenvironment{minilinespace}{
  \small
  \baselineskip=3.0mm
}

\begin{document}
\renewcommand{\theenumii}{\arabic{enumii}}
\renewcommand{\labelenumii}{\textcircled{\small{\theenumii}}}
\def\labelitemi{$\circ$}

\begin{center}
{\LARGE
生活に役立つ物を作ろう

------ マイエプロンを作ろう ------}
\end{center}
\vspace{-1.75zh}
\section{はじめに}
5年生でミシンを使ってナップザックを作ったが、6年生のこの時期にどのような物を作ろうかと考えていたが、児童が6年生になった時から「エプロンはいつ作るの?」(ここ何年かは恒例になっている感はあるが)という声があちこちから聞かれ、児童達もエプロンを作りたいという気持ちが強いのだと思い、エプロンを作ることにした。材料は時間数のことも考え、既製のセットをもとに、一人一人が世界に1つしかない自分だけのエプロン(マイエプロン)と自分で作ったという実感を味わえるような製作をしたいと考えた。そして、物作りの楽しさ、製作した物を生活に役立てる喜びを味わわせ、生活の中で活用しようとする意欲を育てたい。\vspace{-0.75zh}

\section{指導計画        全8時間}
\noindent\hspace{-2zw}
\begin{minipage}{0.65\textwidth}
\begin{enumerate}
\item 製作の計画をたてる\dotfill (1)

\item 製作 \dotfill (6)
\begin{enumerate}
\item 型紙を作る \dotfill $\tfrac{1}{2}$
\item 布にしるしをつけ、布を裁つ \dotfill $\tfrac{1}{2}$
\item 胸の上部\textrightarrow わき\textrightarrow すそ\textrightarrow ひも通しの順に縫う\dotfill 2\\
(三つ折りをする)
\item ポケットを作りエプロンにぬいつける \dotfill 1
\item 自分なりの工夫をする \dotfill 1
\item ひもを通し、アイロンをかけて仕上げる \dotfill 1
\end{enumerate}
\item 発表会 \dotfill (1)
\end{enumerate}
\end{minipage}
\hspace{0.75truecm}
\begin{minipage}{0.35\textwidth}
\includegraphics[width=4.5truecm]{fig1.pdf}
\end{minipage}

\enlargethispage{2truecm}
\section{作品の紹介 工夫の一例}
\noindent
\begin{minilinespace}
\begin{minipage}[t]{3truecm}
\includegraphics[width=3truecm]{P1060271.jpg}
ポケット布の上にタオルかけを縫いつけている
\end{minipage}
\begin{minipage}[t]{3truecm}
\includegraphics[width=3truecm]{P1060272.jpg}
タオルかけとペンさしを余り布で作って縫いつけている
\end{minipage}
\begin{minipage}[t]{3truecm}
\includegraphics[width=3truecm]{P1060273.jpg}
エプロンの裏に六角形のポケット布を縫いつけている
\end{minipage}
\begin{minipage}[t]{3truecm}
\includegraphics[width=3truecm]{P1060274.jpg}
スカートやズボンのように横から手を入れられるポケット布を縫いつけている
\end{minipage}
\begin{minipage}[t]{3truecm}
\includegraphics[width=3truecm]{P1060275.jpg}
ポケット布の上にポケット布を重ねて縫いつけ、中央を縫って半分にポケットを分けている
\end{minipage}
\end{minilinespace}

\section{児童の感想}
%\vspace{12zh}
\begin{itemize}
\item ところどころ難しいこともあったけど、「エプロンを作りたい」という気持ちがあったからこそできたと思う。これから料理をすることがあったら、このエプロンを使っておいしい料理を作りたい。
\item 工夫したところは、ポケットと布を再利用したところです。ポケットは2つに割って2つ入れるところを作った。布の再利用はタオルかけとペンさしとリボンをつけた。みんなと違うエプロンが作れてとってもよかった。
\item 5年の時とは違って大きさや布裁ちも始めからしてすごく時間がかかったけど自分なりによくできたと思う。特にミシンぬいの回数が5年の時より多かった。エプロンの脇の部分はすごく長くてぬうのが難しかった。
\item エプロンを自分で作れるとは思っていなくて途中何回も失敗したけど何とか完成できた。自信もついて、できあがった時はとてもうれしくて、すぐにかけてみた。サイズもちゃんと合っていたのでよかった。
\end{itemize}

\enlargethispage{1truecm}
\section{おわりに}
\begin{itemize}
\item \begin{minipage}[]{5.7truecm}
新聞紙を使って初めてエプロンの型紙作りをした。基本的で一番作りやすい型紙を作った。始めは、「ああ、これが型紙というものか」と半信半疑な感じであったが、できあがってみると各々型紙を自分の体にあてて友達同士見合い、とても嬉しそうで感動している様子も伺えた。
\end{minipage}\hfil
%\vspace{2truecm}
\begin{minipage}[]{7.5truecm}
\includegraphics[width=7.5truecm]{fig2.pdf}
\end{minipage}

\item 縫う箇所はほとんど三つ折りにしアイロンがけをして縫ったが、アイロンの使い方にも慣れ、三つ折りの意味や技法も理解できたようである。長く縫う箇所は慎重に縫っていた。

\item ひと工夫するところでは、使いやすさを考えてポケットの形や縫いつける位置を工夫していた。又、余り布を使ってタオルがけやペンさし、リボン等を作り、縫いつけた。この時間が児童にとっては最も楽しく意欲的に取り組んだものになったように見受けられる。そして、その後の調理実習やクラブ活動の時間にさっそく身につけて作業している子がいた。

\item 一つの物を製作するには、基礎・基本となる知識・技術の習得を積み重ねていくことが大事であると認識した。

\end{itemize}

\end{document}
