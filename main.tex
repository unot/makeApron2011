\documentclass[a4paper,papersize,11pt]{jsarticle}

\usepackage{textcomp}
\usepackage[dvipdfmx]{graphicx}
%\usepackage{otf}
\usepackage{amsmath}

\pagestyle{empty}
%\setlength{\topmargin}{-72pt}
\setlength{\headheight}{0truecm}
\setlength{\headsep}{0truecm}
\setlength{\footskip}{0truecm}

\newenvironment{minilinespace}{
  \small
  \baselineskip=3.0mm
}

\begin{document}
\renewcommand{\theenumii}{\arabic{enumii}}
\renewcommand{\labelenumii}{\textcircled{\small{\theenumii}}}
\def\labelitemi{$\circ$}

\begin{center}
{\LARGE
ミシンで作ろう ナップザック}
\end{center}

%%\section{はじめに}
ミシンの学習をした後、ミシンに慣れ親しめるように生活に役立つ物作りをした。本校では、6年の修学旅行に自分で作ったナップザックを持っていく慣例がある。その1つの大きな目的もあるため、ナップザック作りをした。\vspace{-0.75zh}

\section{指導計画        全8時間}
%\noindent\hspace{-4zw}
\begin{enumerate}
\item 製作の計画をたてる\dotfill (1)

\item 製作 \dotfill (7)
\begin{enumerate}
\item 布にしるしをつける \dotfill $\tfrac{1}{2}$
\item ポケット布にしるしをつけ、布に縫い付ける。(内ポケットになる) \dotfill 1
\item (片方から)\textrightarrow わき\textrightarrow 口あき\textrightarrow 口あきどまり(手縫いで返しぬい)\textrightarrow \\ わき\textrightarrow 口あき\textrightarrow 口あきどまり\textrightarrow 出し入れ口\dotfill 4
\item ひもを通す \dotfill $\tfrac{1}{2}$
\item 自分なりの工夫をする \dotfill 1
\end{enumerate}
\end{enumerate}
%\noindent\hspace{-20zw}

\enlargethispage{2truecm}
\section{内ポケットについて}
\noindent
\begin{minilinespace}
\begin{minipage}[t]{3truecm}
\includegraphics[width=3truecm,height=2.25truecm]{P1070181.jpg}
ポケット布にしるしをつける
\end{minipage}
\begin{minipage}[t]{3truecm}
\includegraphics[width=3truecm,height=2.25truecm]{P1070185.jpg}
ポケット布をまち針でとめる
\end{minipage}
\begin{minipage}[t]{3truecm}
\includegraphics[width=3truecm,height=2.25truecm]{P1070186.jpg}
ポケット布をミシンで縫い付ける
\end{minipage}
\begin{minipage}[t]{3truecm}
\includegraphics[width=3truecm,height=2.25truecm]{P1070280.jpg}
内ポケットがついた
\end{minipage}
\begin{minipage}[t]{3truecm}
\includegraphics[width=3truecm,height=2.25truecm]{P1070281.jpg}
 外側から見ると
\end{minipage}
\end{minilinespace}

\section{児童の感想}
%\vspace{12zh}
\begin{itemize}
\item 始めはむずかしそうだなあと思って緊張もしたけど、だんだん面白くなってきてわくわく感もあり、最後までできて本当によかった。使うのがとても楽しみ。
\item 作る順序がたくさんあったので時間がかかりそうだと思ったけど、やっぱりミシンだと早く完成した。
\item 始めに縫う時、ミシンでこんなに楽しいとは思っていなかったので、できあがってよかった。
\item 作っている時、ミシンの縫っている音が、縫っているという感じで気持ちよかった。
\item 内ポケットは大切な物や小物を入れるのに便利。ばらばらになりにくいのでよかった。収納する場所がふえてうれしかった。
\item 完成して早く背負いたいなあとおもった。明日からでも要る物をすぐに入れたい。そして大切に使いたい。来年の修学旅行も楽しみ。
\item 家でも布を買って1人で作ってみたい。
\item お父さんとお母さんがすごいと言ってくれたからうれしかった。
\end{itemize}

%\enlargethispage{1truecm}
\end{document}
